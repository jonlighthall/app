Teaching an introductory course offers an interesting challenge.  By virtue of the introductory nature of the class, most of the students are unfamiliar with a majority of the material.  In addition to covering new material, the aim of an introductory course is also to present a new way of looking at the word and a new way of approaching problems.  It is all too common for instructors to be out of touch with just how new and confusing an introductory class can be.  This quite naturally can lead to a very stressful experience for the students.\\

While teaching Physics 107, it was my goal to promote learning while minimizing the stress of the students.  Most of the students enrolled in Physics 107 were specifically required to take the class by programs outside of the Physics Department.  For many of the students, this was their first physics class.  It was therefore especially important that the students felt that they have a hand in their learning.  It is this principle that has lead me to develop the format of my class.\\

This also brings me to what I find to be one of the most interesting aspects of my course.  By being able to accurately predict the content of the exams, they are able to study more effectively.
At the time of the exams, the students aren't allowed any outside materials.  And yet they are able to do sophisticated physics problems.  By taking the mystery out of the curriculum, the students are able to concentrate on the core concepts of the course.  As a result, the exams are an opportunity for the students to exhibit what they have learned instead of their ability to look through their notes.
So far, the results speak for themselves.  Over halfway through the semester, the class average is nearly 80\%.  While at the same time we are covering significantly more material than in past semesters.  This is all happening in a class where the majority of the students have never had a physics class before.
\cvwrappersimple{fancy}{\statename}{%
%\clearpage
%%\newpage
%\cvwrapper{fancy}{\statename}{%
%Given below is my research statement.
%}
%\pagestyle{fancy}%
%
\section*{Background}
\addcontentsline{toc}{section}{Background}
%%-----------------------------------------------------------------------------------------------
I have over fifteen years of experience in physics research, specializing in the technical development of new experimental hardware.
Starting in 2002 as an undergraduate, I helped establish thin film production facilities at Michigan State University and Albion College.
As a graduate student at Western Michigan University, the first three years of my research experience focused 
on measurements made at the 
%on two experiments measuring the pair-production branching ratio of the Hoyle state in $^{12}$C using a novel plastic scintillator array. 
% As a part of this work, I learned to operate the 6\,MeV 
tandem Van de Graaff accelerator laboratory at WMU.  %In addition to the typical duties associated with running an experiment---detector setup, data acquisition, and analysis---I also was responsible for designing and machining the detector mounts.% that were used.
%During this time I was also collaborating on a number of experiments at Argonne National Laboratory; these experiments measured transfer reactions on radioactive beams produced in-flight and were measured with an array of double-sided silicon strip detectors.
As an undergraduate and junior graduate student, I was involved in student projects which covered all levels of an experiment from planning to publishing.

I began full-time work at Argonne in %July
 2007 and the last four years of my doctoral thesis work involved the
 development, commissioning, and operation of the Helical Orbit Spectrometer (HELIOS).
 %I worked on nearly every aspect of the development of this new device since the MRI solenoid upon which it is based was delivered to the lab.  One of my first responsibilities was measuring and analyzing a high-resolution field map of the HELIOS solenoid.  I assembled most of the mechanical structures associated with the spectrometer and was involved in the design of many of the mechanical components.
%I designed and constructed the electronics readout, including fabricating most of the cables.
In addition to the substantial technical aspects of this project,
 a major component of my work was the development of computer code. % for the acquisition, calibration, simulation, and analysis of HELIOS data. 
%Since the commissioning of HELIOS, I have been closely involved with every experiment conducted with HELIOS. 
%  \textit{more details here}  
 %a major component of my work was the development of computer code associated with %data acquisition.  
%HELIOS.
I wrote and developed particle tracking simulations, %for optimizing experimental setups.
% and developed a pre-existing Monte Carlo simulation. 
%I also wrote the basis of the
 the data acquisition code, and % that is still in use today and I established and developed
 the calibration and analysis algorithms for data acquired with HELIOS. 
 I also wrote the proposals for the first two HELIOS experiments and published the results of the first experiment.
%After HELIOS was commissioned, % and into my postdoc appointment,
%As a postdoc at Argonne, 
% I continued to collaborate on %    and have been involved with the setup and running of every 
%the setup, calibration, and analysis of HELIOS experiments, as well as %. %since then. %commissioning 
%I have also helped with the  (both online and offline) of several HELIOS experiments.
%A HELIOS-like spectrometer at FRIB is the ideal device for studying transfer reactions with re-accelerated exotic beams.  
%Most of the measurements made with HELIOS have been of transfer reactions with radioactive beams.
%At the same time, I also collaborated on
% a variety of experiments using the Enge split-pole spectrograph.% During these experiments, it was typically my responsibility to set up the detector electronics and data acquisition 
Each semester at Argonne, I supervised and collaborated on undergraduate research projects related to development of HELIOS.

In %October
 2012, as a postdoctoral research fellow, I began work on the commissioning of the Electromagnetic Mass Analysier (EMMA) %, a heavy recoil spectrometer
 at TRIUMF. % where I am working
% in the Nuclear Physics Department in the Physical Science Division under the supervision of Dr.\ Barry Davids.  I have worked with Dr.\ Davids 
 % for nuclear reactions.
 I developed the hardware and software for 
%One of my main responsibilities was the commissioning of
 EMMA's two gas-filled detectors
%and extensive code development %. %Over the last several months
% I wrote and developed code 
%for the acquisition, calibration, simulation, and analysis of data from these detectors. 
%EMMA parallel-grid avalanche counter and ionization chamber. % detector based on two in-beam tests.
%As a part of this work 
and I wrote an extensive safety review of the EMMA gas-handling system. I was also heavily involved with a number of laborious tasks related to the mechanical development of the spectrometer, including the polishing and assembly of  electrostatic components and the installation of radiation shielding.
%; % September 2013--January 2014,
		%designed the detector readout electronics. %January 2013
		%Supervised student projects.
		Before leaving TRIUMF, I authored an experimental proposal for the radioactive beam commissioning of EMMA. Each semester I supervised an research project for an engineering student.
		
		%During my time at TRIUMF,  I have also collaborated on several IRIS experiments and one TUDA experiment, providing online data analysis.
		 %As the installation of EMMA draws near, I have been working on developing the  spectrometer's burgeoning scientific program.	Most recently, I have been writing and reviewing %wrote the 
		%experimental proposals for EMMA's radioactive beam commissioning.
		
		In 2016, I joined the nuclear physics collaboration between Louisiana State and Florida State Universities as a senior postdoctoral researcher. I am the principle investigator on the $^{17}$F($\alpha$,$p$)  measurement made with the Array for Nuclear Astrophysics and Structure with Exotic Nuclei (ANASEN) detector and have collaborated on a number of experiments at FSU, Argonne, and MSU. I have been collaborating with a small team of graduate students on the calibration an analysis of ANASEN data. As the installation of the Yale large-acceptance Split Pole Spectrograph (SPS) is nearly complete, I will be working on the commissioning of the SPS in the coming months.


%%-----------------------------------------------------------------------------------------------
%% Interests
%%-----------------------------------------------------------------------------------------------
%\pagebreak[4]
\section*{Interests}
\addcontentsline{toc}{section}{Interests}
My research interests reflect my research experience, namely nuclear structure and nuclear astrophysics.  More specifically, my immediate 
research interests include %nuclear structure and
 the evolution of single particle states far from stability at the extremes of neutron excess and nucleosynthesis in the classical \textit{r}-process. % and the ``cold'' \textit{r}-process (discussed below).
 In many cases, the study of structure
and astrophysics go hand-in-hand. For example, %such as 
near the $N=50$ and $N=82$ neutron shells,  
an experiment from which 
the energy, spin and parity of nuclear states can be derived may allow for the determination of the 
the single particle structure of the nucleus and serve as an input to 
%states  with the investigations related to the 
$r$-process%. For example, 
network calculations 
, which
 rely on these quantities as inputs. %the energy and spectroscopic factors of single particle states.

The synthesis of approximately half of the chemical elements heavier than zinc % ($A \simeq 70$)
 occurs in the rapid neutron-capture process.
The exact site of the $r$-process is unknown, but the conditions of 
%The elucidation of the site of the %this so-called
 %$r$-process is one of the most important scientific challenges in astrophysics. 
the ``classical'' $r$-process %is a site-free model %approach
% which assumes
are a hot environment ($T> 1.35$\,GK) with neutron densities beyond %$1\times 
$10^{20}$\,cm$^{-3}$. %and
%Classical calculations have been carried out in
%In the waiting-point approximation, equilibrium %is achieved
 %between neutron capture and photodisintegration %$(n,\gamma)\rightleftharpoons(\gamma,n)$ 
 %%in isotopic chains. %is achieved.
%%In this approach, the reaction path can be  calculated for a given temperature and neutron density via the Saha equation.
%% During equilibrium,
%implies that  the neutron capture rates do not play a role %under these assumptions since
%in shaping the final $r$-process abundance curve.
%%because they are balanced by the photodisintegration reaction rates. 
%% In classical calculations, during most of the $r$-process, $n$-capture rates are irrelevant because these reactions are fast enough to establish equilibrium with photodisintegration.
%However, there has been recent interest in so-called
 %``cold'' $r$-process scenarios in which the 
%%at later %\correction{}{r}
 %%times when 
%the temperature and the neutron density 
%are insufficient to
%%drop, % ("freeze-out phase"),
%%equilibrium can no longer be maintained, $n$-capture rates on a number of unstable nuclei have appreciable effects on the shapes of the predicted abundance distributions.
%%Recent work suggests that there may be two sites for producing nuclei previously attributed to a single $r$-process.
%%: %\correction{,}{:}
 %%merging neutron stars and core collapse supernovae. 
 %%While the temperature and neutron density in % in, and entropy in %the former 
 %%neutron star mergers appear to be appropriate for synthesizing the heaviest nuclei, the neutrino-driven wind core collapse supernova hypothesis does not result in the electron fractions and entropies needed to produce significant quantities of nuclei with $A\gtrsim130$.
 %%However, the observations of $r$-process elements in metal-poor stars cannot easily be explained by 
 %%neutron star mergers, which would  occur neither on short enough timescales nor frequently enough to account for the observations \cite{Qian_2000,Argast_2004}. 
%% An alternative core collapse supernova $r$-process scenario has recently been proposed % by \citet{Banerjee_2011,Banerjee_2013} %\correction{Banerjee, Qian, Heger, and Haxton}{\citet{Banerjee_2013}} % 
%%to take place in the He shell of a massive star that has undergone core collapse. 
%%These %so-called
 %%``cold'' $r$-process scenarios
%% are not hot enough to %no longer 
 %reach the ($\gamma,n)\rightleftharpoons(n,\gamma)$ equilibrium. In these scenarios, the $\beta$-decay rates are in equilibrium with neutron capture rates, which %drives the reaction path much further neutron-rich and
 %requires general knowledge of neutron capture rates of neutron-rich isotopes to determine the  $r$-process abundances.
%% free neutrons %or neutrons which are released from the neutron-rich progenitors via $\beta$-delayed neutron emission 
%%can be captured and thus shape the final $r$-process abundance curve.
Direct measurements are impossible for the reactions of interest %in these cases
 because  the required radioactive targets cannot be produced. Instead, transfer reactions such as $(d,p)$ measurements in inverse kinematics are needed to determine neutron spectroscopic factors, spins, parities, and energies of the relevant states so that the neutron capture cross sections can be calculated.
%At the neutron densities of $\sim10^{19}$\,cm$^{-3}$ found in the He shell $r$-process model,
While studies of nuclei relevant to the classical $r$-process may need to wait for facility upgrades, the nuclei synthesized in cold %the He shell 
$r$-process model %, %cold $r$-process 
are closer to the valley of $\beta$-stability % than in the classical $r$-process 
and are %therefore now
within reach %at ISAC from proton-induced fission of a UC$_x$ target. 
of existing facilities.
%The relatively low temperatures found in the He shell imply that ($\gamma,n)\rightleftharpoons(n,\gamma)$ equilibrium is never
%reached and as a result %therefore
% neutron capture cross sections have a large influence on final predicted abundances.
A particularly important ingredient of calculations of the ``cold'' $r$-process %this model is %therefore
is
the neutron capture cross sections of nuclei with $Z>26$ and $50<N<82$. Experimental data for cases around closed shells are vital for ensuring the reliability of structure models needed to extrapolate to unmeasured nuclei. As the level density of nuclei near closed shells is anticipated to be low, statistical model calculations may be unreliable and must be tested. %cannot be trusted.
% of transfer reaction measurements to study the systematic   %Deriving neutron-capture cross sections via neutron-transfer reactions will provide important insight into astrophysical nucleosynthesis.

\rhead{\desctext{\statename}}
%\cfoot{\thepage{}  of \pageref{LastPage}}

%%-----------------------------------------------------------------------------------------------
%% Goals
%%-----------------------------------------------------------------------------------------------
\section*{Goals}
\addcontentsline{toc}{section}{Goals}
In the immediate term, I have a number of ``day one'' research projects that would be perfectly suited for undergraduate involvement either at the summer fellowship, REU, or senior thesis level.
No matter how well-staffed a project is, there is always a need for more manpower.
%At any laboratory there is a tremendous pressure for graduate students to graduate and for postdocs to publish.
At times %Sometimes 
this leaves quite worthy research projects with no one to direct them. There are a number of such projects from my current and previous collaborations that are ready for immediate adoption and would be well suited for undergraduate physics students.
One such project that I would like to work on % immeiately
is the numerical simulation of a number devices at Florida State University accelerator facility. In particular, it would be invaluable to have an up-to-date simulation of the large-acceptance SPS. Current simulations rely on FORTRAN code developed four decades ago. I propose to develop a simulation of the SPS  using the state-of-the-art Geant4 simulation package. This project could be developed as an interdisciplinary study between Physics, Computer Science, and Engineering. Such a simulation would allow potential collaborators to simulate experiments for viability and help aide in detector configuration and analysis. Another similar project would be the 3D electrostatic simulation of the proportional counter of the ANASEN detector using the program Opera. Our current understanding of how the detector works is limiting our analysis efforts and such a simulation would be greatly beneficial. These simulation projects will allow students at Kalamazoo College to contribute to national research efforts with the convenience of working on campus.

In the intermediate term, the imminent commissioning of three devices that I have worked on---the upgraded HELIOS array at Argonne, the EMMA spectrometer at TRIUMF, and the SPS at FSU---will provide a plethora of collaboration opportunities over the coming months and years.
%The  network of collaborators that I have developed at Argonne and TRIUMF provides the opportunity of collaborating on a number of upcoming experiments, especially in light of the imminent commissioning of the EMMA spectrometer and the start of its scientific program. 
Joining in on these projects will offer real world research experience at national laboratories with international collaborations.
%Being involved in the commissioning activities and early operational actives represent a unique opportunity to learn about the basic operation of experimental equipment %is during the initial phase of operation 
%while all those involved are still ``learning the ropes.''
In addition to running experiments, students at Kalamazoo College will be able to lead characterization studies on the new detector systems.
Given the flexibility of a university laboratory, the SPS in particular will be able to provide hands-on research experience to undergraduate students. There are a number of relatively simple scattering experiments that could be carried out by a small undergraduate research team.

In the long term, I plan on being involved with the installation, commissioning, and operation of the solenoidal spectrometer, recently dubbed 
SOLARIS, which has been proposed at Michigan State University. This device is based on the HELIOS spectrometer at Argonne, which I commissioned for by doctoral thesis. While engaged on the HELIOS project, I worked with several undergraduate students on projects which had scopes that ranged from a few weeks to 2--3 years. 
 SOLARIS will be developed to study direct reactions with radioactive ion beams. Its operation is one of the research priorities 
of both the re-accelerated beam facility (ReA) and the upcoming Facility for Rare Isotope Beams (FRIB) at MSU.
Given its proximity, Kalamazoo College is in an excellent position to be a significant contributor to the development of the SOLARIS spectrometer and the research program at MSU. 
}

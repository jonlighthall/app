\cvheadersimple{fancy}{\teachname} {%
\def\secname{Philosophy}
\section*{\hspace{-\parindent}\secname}
\addcontentsline{toc}{section}{\secname}
My teaching philosophy has two main components which reflect what I believe to be the two main reasons students enroll in classes in the first place. %at an excellent institution like Albion College.
 My view in this regard is enlighten by the time I spent at my alma mater, Albion College. I know what a special experience a small liberal art college can provide and I know of no better learning environment. 
The first component of my teaching philosophy is the importance of personal instruction.
 Students are enrolled in % class 
a school like Albion College because they expected to have direct interaction with the instructors and they value that interaction. Recognizing and respecting this choice is key to creating a space where students can learn.
I make it a priority to be available to students; before and after class and during office hours. Referring to my teaching evaluations (included elsewhere), my students and the faculty found my personal instruction and availability as one of my greatest strengths.

The second component of my teaching philosophy is based on the fact that students have enrolled in class largely as a means to an end. Although that end may be pure academic enrichment, students typically enroll in a particular class because it is specifically required as a part of an educational program. 
This may seem like an obvious point of view, but it is not always considered.
When I taught Physics 107 at Western Michigan University, the class was a prerequisite for the
Aviation Flight Science, 
Speech Pathology and Audiology, 
and 
Earth Science
programs. I went to each of these outside departments and asked what topics the students were expected to learn from having taken my class. Apparently previous instructors hadn't even considered this. For example, previous sections of the class did not cover sound waves or aerodynamic lift, even though students requiring those topics made up two-thirds of the class.
It is fundamentally important to me that students receive the tools they need from the class before moving on.

%
\def\secname{Goals}
\section*{\hspace{-\parindent}\secname}
\addcontentsline{toc}{section}{\secname}
The teaching experience I have is in teaching an introductory course and introductory labs.
Teaching introductory topics offers an interesting challenge.  By virtue of the introductory nature of the class, most of the students are unfamiliar with a majority of the material.  In addition to covering new material, the aim of an introductory course is also to present a new way of looking at the word and a new way of approaching problems.  It is all too common for instructors to be out of touch with just how new and confusing an introductory class can be.  This quite naturally can lead to a very stressful experience for the students. This phenomenon is not unique to introductory courses, as any new material could be intimidating for students and met with trepidation.

While teaching Physics 107, it was my goal to promote learning while minimizing the stress of the students.  Most of the students %enrolled in Physics 107 
were %specifically 
required to take the class %by programs outside of the Physics Department.  F
and for many students, it was their first physics class.  It was therefore especially important that the students felt that they have a hand in their learning.  It is this principle that lead me to develop the format of my class.
%This also brings me to what I find to be one of the most interesting aspects of my course.  
The homework assignments followed the lectures and important assignments were reviewed in class. Weekly quizzes reflected the homework and monthly exams were an accumulation of the quizzes.
At the time of the exams, the students aren't allowed any outside materials. %; and yet they are able to solve sophisticated physics problems.
As a result, the exams are an opportunity for the students to exhibit what they have learned instead of their ability to look through their notes.
Taking the mystery out of the curriculum by being able to accurately predict the content of the quizzes and exams, students are able to study more effectively and 
concentrate on the core concepts of the course.
%So far, the results speak for themselves.  Over halfway through the semester, the class average is nearly 80\%.  While at the same time we are covering significantly more material than in past semesters.  This is all happening in a class where the majority of the students have never had a physics class before.%
My technique was demonstrated successful as the the class average was nearly 80\% and I had outstanding evaluations from my students and the faculty.

%While I don't have a specific preference for teaching assignments, 
Given the opportunity, 
I would be interested in developing a number of special topic courses that I wish I had had access to either as an undergraduate or a master's-level graduate student. In particular, I would really enjoy teaching a class on radiation detection and measurement. This class could cover radioactive decay, radiation detector types, readout electronics, and computer acquisition. This class could serve as a prerequisite or corequisite for the research experience program I would develop to engage students in research programs at laboratories around North America.
There are a number of practicum courses 
that I would like to %could be developed 
develop for software packages such as: % Another class I would be interested in developing 
\LaTeX, the markup language used to create this document and the standard for scientific publication; Git, a version control system useful for code and manuscript development; and ROOT, a data visualization program used in nuclear and particle physics.

\def\secname{Experience}
\section*{\hspace{-\parindent}\secname}
\addcontentsline{toc}{section}{\secname}%
%\note{Statement of teaching philosophy and goals that includes a statement on experience working with underrepresented students and engaging issues of diversity and inclusion in pedagogical approaches}%
%That is why I encourage my students to treat my lectures as a discussion group.
In addition to my experience as a course instructor, detailed above, I have five years of experience as a laboratory instructor.
As a teaching assistant at WMU, I noticed there  was an issue with a lack of student engagement during the laboratory sections. I was expected to help the students if they ran into trouble with the lab equipment, but otherwise the students were supposed to follow the lab manual. One of the problems was that often the lab section was out of sync with the lecture course. To address the problem, I developed short introductory lectures and wrote supplemental materials for each topic covered in the lab. Grades and student evaluations for the course improved significantly after I adopted this approach.

Over the past fifteen years, I have worked with collaborators from every corner of the globe representing an immense diversity of cultural backgrounds.
The students that I taught at WMU, both in labs and in class, came from diverse academic backgrounds each accustomed to different pedagogical paradigms.
At Argonne, TRIUMF, and FSU, I have supervised and mentored students from high school, college, and graduate school.
Through all of these interactions, one thing has remained constant: teaching is a social dynamic. The key to engaging students and colleagues is to create a personal connection based on mutual respect.

%The successful candidate will be expected to be able to develop and teach undergraduate labs. 
%
%All of the students expected to get something different out of the class and all of the students 
}
\cvfootersimple{}
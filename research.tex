\cvwrapper{fancy}{\statename}{%
My research interests mirror my research experience, namely nuclear structure and nuclear astrophysics.  More specifically, I am interested in the evolution of single-particle states at the extremes of neutron-rich nuclei; such as near the $^{132}$Sn doubly-magic shell closure.  I'm also interested in the properties of nuclei near or on the astrophysical $r$-process path.\\

I have about ten years of experience in physics research, the majority of which is related to the mechanical development of new experimental hardware.  As an undergraduate, I helped establish the thin film laboratory at Albion College.  I built and characterized the vacuum system on the thin film deposition chamber.  As a graduate student, the first three years of my research experience focused on a pair of measurements of the branching ratio of the Hoyle state in $^{12}$C using a plastic scintillator array at Western Michigan University.  As a part of this work, I learned to operate the 6\,MeV tandem Van de Graaff accelerator at WMU.  In addition to the typical duties associated with running an experiment, e.g. NIM electronics setup, data acquisition and analysis, I also was responsible for designing and machining the detector mounts that were used.  During this time I was also involved with a number of experiments at Argonne National Laboratory; these experiments utilized radioactive beams produced in-flight and were measured with an array of double-sided silicon strip detectors.\\

The last four years of my doctoral thesis work involved the %development, 
commissioning and operation of the Helical Orbit Spectrometer (HELIOS) at Argonne.  I worked on nearly every aspect of the development of this new device since the MRI solenoid upon which it is based was delivered to the lab.  One of my first responsibilities was measuring and analyzing a high-resolution field map of the HELIOS solenoid.  I assembled most of the mechanical structures associated with the spectrometer and was involved in the design of many of the mechanical components.
I designed and constructed the electronics readout, including fabricating most of the cables.
%  \textit{more details here}  
In addition to the mechanical and technical aspects of this project, a major component of my work was the development of computer code associated with %data acquisition.  
HELIOS.  I wrote an analytical particle track simulation for optimizing experimental setups and developed a pre-existing Monte Carlo simulation.  I also wrote the basis of the data acquisition code that is still in use today and I established and developed the calibration and analysis algorithms for data acquired with HELIOS.  Most of the measurements made with HELIOS have been of transfer reactions with radioactive beams.  I was the Principle Investigator for the first two HELIOS experiments and have been involved with the setup and running of every HELIOS experiment since then. %commissioning 
I have also helped with the calibration and analysis (both online and offline) of several HELIOS experiments.
%A HELIOS-like spectrometer at FRIB is the ideal device for studying transfer reactions with re-accelerated exotic beams.   of transfer reaction measurements to study the systematic   %Deriving neutron-capture cross sections via neutron-transfer reactions will provide important insight into astrophysical nucleosynthesis.
}